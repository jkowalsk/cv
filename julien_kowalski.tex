\documentclass[french, a4paper]{customcv}

\bibliography{biblio.bib}

%----------------------------------------------------------------------------------------
%	 PERSONAL INFORMATION
%----------------------------------------------------------------------------------------
% If you don't need one or more of the below, just remove the content leaving the command, e.g. \cvnumberphone{}

\cvname{Julien Kowalski}
\cvjobtitle{ Ingénieur en cryptographie } 

\cvlinkedin{julien-kowalski-002b99105}
\cvgithub{jkowalsk}
\cvnumberphone{(+33) 6 30 75 10 88} % Phone number
\cvsite{} % Personal website
\cvmail{kowalski.julien@gmail.com} 

%----------------------------------------------------------------------------------------
%	 Skills definition
%----------------------------------------------------------------------------------------
\skills{
  \resizebox{\textwidth}{!}{
    \centering 
    \smartdiagram[bubble diagram]{
      \textbf{Sécurité} \\ \textbf{de l'information},
      ~~~~~~\textbf{Cryptographie}~~~~~~ \\ spécifications,
      ~~~~\textbf{Cryptographie}~~~~ \\ implémentation,
      ~~~~~~~~~~~~\textbf{Critères}~~~~~~~~~~~~ \\ \textbf{Communs},
      \textbf{Architecture}\\ \textbf{logicielle},
      ~~~~\textbf{Architecture}~~~~\\ \textbf{sécurité}
    }
  }
}

% skill bars
\skillbars[Développement]{%
  {Rust $\bullet$ Python / 3},%
  {C $\bullet$ C++ $\bullet$ Java  / 4.5},% 
  {JavaCard $\bullet$ \large \LaTeX / 5.5}%
}

%----------------------------------------------------------------------------------------
%	 Education part
%----------------------------------------------------------------------------------------
\education{
  \textbf{Stages (CFSSI)} \\
  Cryptologie $\bullet$ méthode EBIOS $\bullet$ Critères Communs
  
  \textit{2004} -- \textbf{DESS Sécurité de l'information} \\
  Université de Limoges. \\
  Mention Bien
  
  \textit{2003} -- \textbf{Licence d'informatique} \\
  Université d'Evry Val d'Essonnes. \\
  Mention Bien
  
  \textit{2000} -- \textbf{CAPES de Mathématiques}
  
  \textit{1999} -- \textbf{Maîtrise de Mathématiques} \\
  Université d'Evry Val d'Essonnes. \\
  Mention Assez Bien
  
  \textit{1998} -- \textbf{Licence de Mathématiques} \\
  Université d'Evry Val d'Essonnes. \\
  Mention Bien
  
}


\begin{document}
\makeprofile
%----------------------------------------------------------------------------------------
%	 EXPERIENCE
%----------------------------------------------------------------------------------------
\section{Expérience}

\begin{cventry}{2009 -- }{Ingénieur en cryptographie}{\href{https://www.ercom.fr/}{Ercom}}
  \begin{itemize}
    \item  Architecture logicielle et de sécurité;
    \item Spécifications et développements cryptographiques des produits Ercom~:\linebreak
          Cryptosmart; Cryptopass et Cryptobox
    \item Responsable des \textbf{évaluations Critères Communs}
          \begin{itemize}
            \item Carte Cryptosmart (\cite{anssi.2012/71} ; \cite{anssi.2016/69})
            \item Cryptobox \cite{anssi.2018/23}
          \end{itemize}
    \item \textbf{Innovation}
          \begin{itemize}
            \item IBE, ABE, pairings, searchable encryption, anonymat.
            \item Dépôt de brevets (\cite{pat.save}, \cite{pat.card.auth})
          \end{itemize}
    \item Relations avec l'ANSSI
  \end{itemize}
\end{cventry}

\begin{cventry}{2009}{Consultant en cryptographie}{Alten}
  En mission pour Thalès Communication
  \begin{itemize}
    \item Participation aux spécifications du projet
    \item Définition de l'architecture de clefs
  \end{itemize}
\end{cventry}

\begin{cventry}{2004 - 2009}{Ingénieur crypto-mathématicien}{Ministère de la Défense}
  \begin{itemize}
    \item \textbf{Développement de produits de sécurité:}
          \begin{itemize}
            \item \'Etude du besoin et des menaces associées
            \item Conception et développement de solutions de sécurité
          \end{itemize}
    \item \textbf{Analyse de solutions de sécurité : }
          \'Etudes de sécurité inspirées des Critères Communs et d'EBIOS
    \item \textbf{\'Evaluation de produits civils :} Suivi technique d'évaluation de produits de sécurité
  \end{itemize}
\end{cventry}

\begin{cventry}{2004}{Stage de fin d'études}{Ministère de la Défense}
  Bases de Gr\oe bner : application  à la cryptanalyse de registres à décalage linéaire filtrés.
\end{cventry}

%\twentyitem{<dates>}{<title>}{<location>}{<description>}
\subsection{Enseignements}
\cventrynobody{2002 -- 2003}{Colleur en mathématiques spéciales}{\'Education Nationale}
\cventrynobody{2000 -- 2002}{Professeur de Mathématiques}{\'Education Nationale}


\end{document}
